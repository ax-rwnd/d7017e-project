In the very beginning of the project, most communication was done in full-group meetings. This format worked because the questions discussed were relevant to everyone involved. Over time, it was found that having everyone participate in meetings added a lot of overhead and so it was decided to split the project into subgroups of approximately 5 members.

For instance, during the pre-study, there was one group that conducted interviews, another group that studied the literature around gamification and so on. By dividing the project group into smaller parts with leaders for each group, some members could be made responsible for managing inter-group communication and management, while the others focused on their work. Above all groups, a project leader was elected to lead the project forward, manage groups and assist the groups in tough design decisions. %Above all groups, a manager was elected to call meetings, manage communication with the project owner and to handle common tasks related to human resources.

%All of the main task during the project were split into groups. During the project it was laborated on how many number of groups would be the most efficient for the project. The working model that was found was to have three groups with 4-5 members each as long as there were enough tasks in different sections for three groups. Once time in a week all members met and discussed the previous week and the upcoming week. This became a meeting with 15 people that ended up talking over each other which felt ineffective. This was solved by minimizing the meeting members to three-four people: the project leader along with one group leader each from the main groups. The leaders were democratically chosen in each active group and their task was to share with the other group leaders what they have done during previous period and what will they accomplish during next period. The groups leaders represented the groups and had as responibillity to share the discussions from the meetings to their groups.
