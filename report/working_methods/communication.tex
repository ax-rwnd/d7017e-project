\slack{} was used as a common platform for communication, this allowed the project members to quickly share issues, ask questions, call to meetings and post funny- or interesting images. Particularly useful was the ability to create channels and group chats, this enabled both the larger groups to communicate in their own spaces and smaller task-forces to quickly assemble to discuss bug-fixes.

However, the members were still encouraged to attend the project-room for every-day programming sessions, this was done to ensure that everyone communicated with one another and was able to get help with whatever issues they had. Furthermore, this enabled pair-programming and extended design-decisions to be made by multiple participants.
%For the purpose of communication within large groups this tool worked well.


%To share information between all of the group members \slack{} a shared workspace where everyone can write to each other. It is possible to communicate thorough dedicated channels for different topics but also direct messages between each other. \slack{} has support for smart phones, web application and a application for desktop. There is possibility to share code snippets and documents which is very suitable for a computer since project.

%Lot of communication between members were also done by talking to each other in the project room or if someone was on remote calls was made by phone or \skype{}.
