To plan the project, an agile workflow was adopted, this meant dividing the work into sprints, where the end of each sprint represented a project presentation. In each presentation, some progress that was not in the solution the week prior was presented, which became the implicit milestones of the project.

Sprint planning was done using the \taiga{} taskboard, by using a web-based tool, most planning could be done remotely instead of on a physical taskboard.Furthermore, participants didn't need to visit the project-room to see- or update their tasks, instead they could edit the tasks remotely from their own computers.


%Agile work flow was used during the project. This was done by dividing the weeks into sprints. A sprint was expected to be one week but sometimes sprints become longer due to heavy workload. Every sprint was planned before the current sprint was ended. The group leaders planned the sprints, one from each current sub group along with the project leader. \taiga{}, a sprint planning tool kept track of the backlog and different task assigned to individual group members. This showed which task was completed and which needed to be helped with. 



