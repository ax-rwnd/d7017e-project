Since the project was done as a double-period course at \LTU, $\SI{400}{\hour}$ per person was planned for the project over four months, this would amount to $400\cdot15=\SI{6000}{\hour}$ in total. To plan these hours, an agile workflow was adopted. This meant dividing the work into sprints, where the end of each sprint represented a project presentation. In each presentation, some progress that was not in the solution the week prior was presented, which became the implicit milestones of the project.

For instance, in the first presentation the architecture was to be presented and a simpler technical demonstration was to be done. So, a proof-of-concept wherein python code was to be tested using a very simple user interface was designed. The idea was to include some mock data from the backend as well, but at the day of the presentation, the API was not ready enough. Right after the presentation, a new goal to have the database setup and storing the progress of the corrected tests was planned for the next week.

\subsubsection{Taiga Sprint Planner}
Sprint planning was done using the open-source \taiga{} taskboard. By using a web-based tool, most planning could be done remotely instead of on a physical whiteboard. An idea was that it would allow participants to do their work remotely and not have to bother with visiting the project room. Taiga was chosen because it offered a free and gratis environment with little hassle for public projects, which seemed like a good fit for this project since everyone was eager to get started.

Apart from \taiga{}, Trello was evaluated, but deemed too simple for a project of this size, furthermore some pay-to-use tools such as TODO: was looked at, but nobody wanted to pay for the service.
