\chapter{Discussion}
\section{General} 
The projects outcome can be considered successful. The system that has been delivered has the ability to test code and give feedback to students and has elements of gamification. The gamification part can be further developed but we have built a good basis for it. The gamification and quick feedback that is readable is very important when someone is new at coding. The standard outputs from a compiler can be difficult to understand if you're new. The availability to see if the code written is correct and and matches the teachers output is important since sometimes feedback from the teacher or lab supervisor takes long time and an assignment can take long time to complete. Many new students are afraid to ask dumb questions since they don't want to be classified as bad. A anonymous question handler would be suitable so that everyone can ask questions and answer them completely anonymously. The major difficulty here is to choose which parts should be included. Everyone that was asked thought different and we choose the parts based on repetitive answers from teachers and our own preferences. We also thought that when a teacher creates a course, the teacher should have the ability to choose what gamification parts should be included for the specific course. This may, based on the prestudy, be a bit bad since it's shown that when a reward that is usually given for a something is removed, the motivation decreases. There is still a lot of things that could be implemented from gamification and from the question form and hopefully someone will continue implementing features and test this system on students. There is a large part left which isn't covered and that is to set up a workshop with students were they test and gives feedback about the system. It was planned to be done but due to prioritizing functionality this was left.

\subsection{Work}

During the project there has been many confusions what should be done, why and how. We were 15 persons in the project although the project was dimension for approximately 7-9 persons. Non of the project members have ever worked in a such large group so it was difficult to start the project due to some people took more place than others. The projects outcome was unclear for a very long time and projects specification and requirements should be more detailed, this for help the project to start faster. The prestudy part took too long time, there was many parts to cover and required. We was told to have a workshop but the workshop was held when the implementation should started so the project stayed up for a while waiting for the result of the workshop. Due to another heavily course simultaneously in period 1 most of the project members prioritized that course and that meant that some task wasn't completely finished that time it should be done and become overhanging to next sprint. 
Since there was 15 persons in the project it become some groupings and the communicating lacked between the groups. This was a hard problem to solve and the only way to solve was to encourage everyone to sit in the classroom and talk to each other. Two different meeting methods were tested, the whole group and leaders from groups. When the whole group was collected the meeting become more that one or few talked and the other sat quiet and the meeting become long and nothing felt concluded after the meetings. So we changed that every group selected a leader that went to leader meeting. this felt good but the leaders didn't bring all information to the meeting and didn't always remember everything from them so that the information the group got from their leaders may not be complete. So the conclusion is that the projects groups should be smaller, about 7-9 persons in a project is more suitable since the persons taking this course mostly doesn't have experience working in projects and more less in large groups.


\section{Conclusion}
Gamification has become a large part of motivation. By getting rewards for performed works the motivation and satisfaction increases. Gamification with immediate feedback in programming is fairly new since learning programming has grown and spread to lower ages, a tool for learning programming quickly is needed.  
The platform provided here is dynamic in that way that there is no limitation in the difficulty in the assignments, the creator of the assignment is that one which sets the level. By that said, even since the target is university students, the platform is also applicable to younger students. There are endless choices for new gamification parts that can be implemented, only the imagination puts a stop to which parts can be implemented. The platform delivered is an initial working product that is usable but can be expanded to fairly low cost. 

\section{Acknowledgements}
We would like to express our gratitude to the teachers that have been available for interviews and attended the workshop that has helped us understand different gamification aspects. We would like to thank Prof. Peter Parnes for the platform idea and the tuition during the project.
