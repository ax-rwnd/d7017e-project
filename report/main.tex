\documentclass[a4paper,12pt]{report}

% input/text packages
\usepackage[utf8]{inputenc}
\usepackage[english]{babel}
\usepackage{amssymb}
\usepackage{calc}

% graphical/layout packages
\usepackage[top=1in, bottom=1.5in, left=1in, right=1in]{geometry}
\usepackage{graphicx}
\usepackage{subcaption}
\usepackage{float}
\usepackage{enumitem}
\usepackage{multicol}
\usepackage{rotating}
\setlist[itemize]{noitemsep, topsep=0pt}
\setlist[enumerate]{itemsep=0pt, topsep=0pt}

%Set graphics path
\graphicspath{ {img/} }

% Horizontal rule with negative spacing
\newcommand{\spacerule}[1]{\vspace{-.66\baselineskip}\rule{#1}{.4pt}\par}

% Group members together to keep title code clean
\newcommand{\membergroup}[4]{%
    \begin{member}{\raggedright}#1\par\spacerule{\widthof{#1}} #2\end{member}\hfill%
        \begin{member}{\raggedleft}#3\par\spacerule{\widthof{#3}} #4\end{member}%
            \par\vspace{\baselineskip}}

% Simple minipage for listing project members
\newenvironment{member}[1]{\begin{minipage}[t]
    {.3\textwidth}#1}{\end{minipage}}

\usepackage{xcolor}
\usepackage[colorlinks = true,
            linkcolor = blue,
            urlcolor  = blue,
            citecolor = blue,
            anchorcolor = blue]{hyperref}

\def\LTU{Luleå University of Technology}
\def\taiga{\href{https://taiga.io}{taiga.io}}
\def\techio{\href{https://tech.io}{tech.io}}
\def\sockr{\href{https://github.com/ArmedGuy/Sockr}{Sockr}}

\def\nodejs{Node.JS}


\begin{document}

\begin{titlepage}
    \vspace{2\baselineskip}

\rule{\linewidth}{.02cm}

\vspace{\baselineskip}

\begin{minipage}{.45\linewidth}
\raggedright\Huge Gamified Programming Platform
\end{minipage}
\hfill
\begin{minipage}{.45\linewidth}
\raggedleft\Large A project in Information and Communication Technology
\end{minipage}

\vspace{-2\baselineskip}

\vfill

\Large\it\LTU

\vspace{-.33\baselineskip}
\rule{\widthof{\LTU{}}}{.02cm}
\vspace{.66\baselineskip}

\begin{minipage}{.45\linewidth}
\Large\it\SRT
\end{minipage}

\vfill

\begin{center}
\normalsize\url{https://github.com/ax-rwnd/d7017e-project.git} \\
\normalsize\url{https://www.youtube.com/watch?v=NiJk_54Qsck}
\end{center}

\vfill

\begin{minipage}[b]{.45\linewidth}
{\large Supervised by\\ Peter Parnes}
\vspace{\baselineskip}
\end{minipage}
\hfill
\begin{minipage}[b]{.35\linewidth}
\includegraphics[width=\linewidth]{img/LTU_logo.png}
\end{minipage}

\end{titlepage}

 \begin{abstract}
 This report presents how gamification could be applied when learning programming. The key aspects of gamification were investigated by conducting interviews with teachers at Luleå university of technology and a theoretical study. The produced platform shows how some parts of gamification can be used when learning programming. The platforms primary target is university students taking computer science courses at Luleå university of technology. The product is a platform for increasing learning by gamification. In the platform it is possible for teachers to create courses with gamification elements such as an adventure map, a leader board, a progress bar and badges. The teacher can also create assignments and choose how the assignment should be tested, input/output test and/or Lint-test the code. 
The student can join courses, solve assignments, get feedback if the programming assignment have been correctly solved and earn rewards for increased motivation.
 \end{abstract}

\tableofcontents{}

% NOTE: \addtocontents need to keep this format for the toc to be properly columnized
\addtocontents{toc}{\protect\begin{multicols}{2}}

 \chapter{Introduction}

\section{Background}
At \LTU\ there are many courses where programming is a major subject. Students sometimes face difficulties in knowing whether they have programmed an assignment correct or not until a lab supervisor or examine checks the code. This dependency on human interaction in general lead to slow feedback. Thoughts, reflections and memories regarding the assignments are often forgotten due to the time passed between writing the code and receiving feedback on the code.

Apart from this, students frequently find that their studies in programming are useless because they do not undestand where- and how this knowledge can be applied. In other words, they lack the context they would need to actually motivate themselves to learn programming. By providing context, the students can have a better understanding of how they can apply their knowledge until they actually encounter the real-life uses.\\
\\
To shorten the time required for feedback and make assignments more fun, an interactive web tool for coding practice was requested. There are many different applications on the internet that already does various parts of this, but none that are freely available and suitable for usage in connection with university courses.

None of the tools that were investigated were able to fill all the functionallity that teachers may want to use to create exercises. The tools were either build like games or had very little- to no pedagogical parts and the few pedagogical parts were quite boring. It also turned out that many of the tools were too complicated for teachers to implement in existing courses. The tools would need to be simplified to be useful or too much time would be spent on maintenance.



\section{Pre-study}
\subsection{Interviews}

\subsection{Literature studies}

\subsection{Workshop}
\subsubsection{Planning}
As part of the pre-study, it was planned from the start of the project to host a workshop. A workshop is an interactive session in which participants work together on solving issues through discussion and brainstorming\cite{workshop}. 

A subgroup of the project members were responsible for planning and hosting the workshop. The planning consisted of brainstorming sessions and a consulting session with the project owner for advice on how to manage a workshop. An important question that was discussed in the brainstorming process was what the actual purpose of the workshop should be. It was decided on holding the workshop with the second out of two purposes:

\begin{list_type}  
    \item Try out game mechanics on the participants to see which mechanics should be included in the product.
    \item Find out from the participants what features the product should have and mockups of the suggested system. 
\end{list_type}

To elicit information in the workshop in accordance to the purpose of the workshop, three workshop scenarios were designed. The idea behind the scenarios is to have the participants discuss freely on certain issues that are relevant to the product. Short description of the three scenarios (for full information see \ref{chapter:workshop}):

\begin{list_type}  
    \item \textbf{Scenario 1:} Put the participants in the role of a student having to complete complete cumbersome programming assignments, think of a system that would make this more appealing.
    \item \textbf{Scenario 2:} Put the participants in the role of a teacher which years later holds the same course. Further develop and/or limit the system.
    \item \textbf{Scenario 3:} User-interface mockups of the system.
\end{list_type}
    
\subsubsection{Hosting}
Invitations were sent to students, teachers and employees at the University Pedagogy Centre, all from Luleå University of Technology. There were 12 participants in total, five of which were moderators. The participants were randomly split into two groups. They received print-outs of the scenarios and had limited time to discuss and write down ideas on post-it notes and white-boards, which would later be presented.

\begin{figure}[H]
\centering
\includegraphics[width=0.8\textwidth]{img/gppinpictures/workshop1_resized.jpg}
\caption{Workshop in action: 1}
\label{fig:workshop1}
\end{figure}

\begin{figure}[H]
\centering
\includegraphics[width=0.8\textwidth]{img/gppinpictures/workshop2_resized.jpg}
\caption{Workshop in action: 2}
\label{fig:workshop2}
\end{figure}

\begin{figure}[H]
\centering
\includegraphics[width=0.8\textwidth]{img/gppinpictures/workshop3_resized.jpg}
\caption{Workshop in action: 3}
\label{fig:workshop3}
\end{figure}

\subsubsection{Results}
Summarized results from the workshop:

\begin{itemize}
    \item Gamification elements will motivate students to perform the assignments.
    \item Social support, encourage students to cheer for one-another through collaboration.
    \item Anonymous questions so students doesn't have to be afraid to ask questions. Should generate statistics for the teacher.
    \item More direct contact between teachers and students, live-chat.
    \item Fast feedback for students when doing assignments. 
    \item Personality tests to customize gamification elements and/or UI.
    \item Students correct assignments completed by other students, and create assignments for one another.
    \item An internal knowledge-bank for the system, similar to \href{https://stackoverflow.com/}{Stackoverflow}, where students can find help with certain courses and/or assignments.
\end{itemize}

\begin{figure}[H]
\centering
\includegraphics[width=0.8\textwidth]{img/gppinpictures/Grupp2_Workshop_mockup.jpg}
\caption{Workshop in action: 4}
\label{fig:workshop4}
\end{figure}


\section{Use-Cases}
In the system there should be two different roles, teacher and student.
The teacher is able to create courses, assignments and invite students to the courses.
The teacher will create a course by fill out the description and the name of the course. Then it's possible to choose which gamification parts that should be included in the course. The visibility of the course is chosen by two options public and auto join. When public, the students can see that the course exist and if auto join is chosen the students can freely join the course otherwise they have to request access. When a teacher creates a assignment the teacher fills out the name of the assignment and chooses between LINT-test and input/output test that should be applied on the assignment. The teacher can then write the assignment in with markdown for better structure and easier reading. It is now possible to create badges where the teacher chooses which badges should be applied to an assignment.\\

A student can be invited to a course or join courses. The student that want to solve an assignment can choose the assignment from a list of assignments on the course page or choose the assignment in the adventuremap if it has been selected by the teacher. The student can see the progress of solved assignment and rewards in form of badges, leader board and progressbar. When solving an assignment the student writes the solution and then submits it. The student will get direct feedback with a reward if the solution is correct otherwise the feedback will be a text showing which tests has failed.


\section{Existing Tools}
To meet the requirements and goals of the project it was clear that a platform was needed that could compile and run code. An initial suggestion from the project description was to use \techio{}, which is a collaborative platform to share coding assignments through open-source ``playgrounds''. The platform seemed to match the needs of the project and the project owner had been in touch with the developers, but no API access had been guaranteed.
\begin{figure}[ht]
    \begin{subfigure}{.45\linewidth}
        \includegraphics[width=\linewidth]{img/techio_game.jpg}
        \caption{One of the games.}
    \end{subfigure}
    \hfill
    \begin{subfigure}{.45\linewidth}
        \includegraphics[width=\linewidth]{img/techio_handson.jpg}
        \caption{Coding in \techio.}
    \end{subfigure}
    \caption{Example pictures from \techio.}
\end{figure}

After some investigation into the platform and discussion with the developers, it was found that \techio{} does not have---and will not get---any open APIs. Thus, the only possibility of using their system was to  embed ``code snippets'' from their site into the one that was to be built. Because of this shortcoming, \techio{} was discarded.

In contrast to \techio{}, a student at \LTU{} has developed an open platform called \sockr{} for hosting programming ladders. However, while the solution was more interesting from the licensing point of view, \sockr{} didn't have any documentation and required too much refactoring to be of use for the project.


 \chapter{Working Methods}

\section{Group Structure}
In the very beginning of the project, most communication was done in full-group meetings. This format worked because the questions discussed were relevant to everyone involved. Over time, it was found that having everyone participate in meetings added a lot of overhead and so it was decided to split the project into subgroups of approximately 5 members.

For instance, during the pre-study, there was one group that conducted interviews, another group that studied the literature around gamification and so on. By dividing the project group into smaller parts with leaders for each group, some members could be made responsible for managing inter-group communication and management, while the others focused on their work. Above all groups, a project leader was elected to lead the project forward, manage groups and assist the groups in tough design decisions. %Above all groups, a manager was elected to call meetings, manage communication with the project owner and to handle common tasks related to human resources.

%All of the main task during the project were split into groups. During the project it was laborated on how many number of groups would be the most efficient for the project. The working model that was found was to have three groups with 4-5 members each as long as there were enough tasks in different sections for three groups. Once time in a week all members met and discussed the previous week and the upcoming week. This became a meeting with 15 people that ended up talking over each other which felt ineffective. This was solved by minimizing the meeting members to three-four people: the project leader along with one group leader each from the main groups. The leaders were democratically chosen in each active group and their task was to share with the other group leaders what they have done during previous period and what will they accomplish during next period. The groups leaders represented the groups and had as responibillity to share the discussions from the meetings to their groups.


\section{Communication}
To share information between all of the group members \slack{} a shared workspace where everyone can write to each other. It is possible to communicate thorough dedicated channels for different topics but also direct messages between each other. \slack{} has support for smart phones, web application and a application for desktop. There is possibility to share code snippets and documents which is very suitable for a computer since project. \\
Lot of communication between members were also done by talking to each other in the project room or if someone was on remote calls was made by phone or \skype{}.

\section{Planning}



Agile work flow was used during the project. This was done by dividing the weeks into sprints. A sprint was expected to be one week but sometimes sprints become longer due to heavy workload. Every sprint was planned before the current sprint was ended. The group leaders planned the sprints, one from each current sub group along with the project leader. \taiga{}, a sprint planning tool kept track of the backlog and different task assigned to individual group members. This showed which task was completed and which needed to be helped with. 





 \chapter{Design}

\section{Architecture}
\begin{figure}[H]

\includegraphics[scale=0.7]{img/SystemA2.png}
\caption{A simple sketch over the system parts and how they are connected.}
\end{figure}

\section{Frontend}
The frontend is built using the JavaScript framework Angular 4. Other large frameworks were also considered but in the end Angular seemed like the best choice. It is used in a lot of web applications so it's easy to find tutorials, there is a lot of support for it, there are many existing packages that can be imported and it is used a lot in the industry.

The site consist of a number of components. These include the different pages in the application but also other visually large parts such as the gamification elements. There are services that keep track of information used by different components and also one that is in charge of the backend calls.

The CSS library Bootstrap was used to get a clean and consistent design. To achieve a distinct and personal look a color scheme was decided upon. For this the dark blue color of LTU was used as a base.

\subsection{Authentication}
\subsubsection{Login}
To authenticate yourself on login you are by default prompted to sign in with LTU's weblogon CAS (central authentication service). 
This provides you with a ticket as an URL parameter and redirects you to path: \url{https://frontend\_ip/Auth} which in turn contacts backend with said ticket to check its validity. If validated the backend responds with a status code of 200 OK and a JSON object containing authentication tokens for the user which are saved in local storage. In any other case the user receives status code 401 unauthorized and is prompted to login again.
\begin{figure}[hb]
    \centering
    \includegraphics[width=\textwidth]{login.png}
\end{figure}

\subsubsection{Auth interceptor}
The main purpose of this interceptor is to modify all requests by setting the authentication field in the header of the request for authenticated users. More than this it collects failed requests in the case of token expiration and tries to refresh the token to retry all failed requests automatically and stay logged in without the user being aware of affected.

\subsection{User page}
Since the user page acts as the front page of the site it should display a good overview to the user. Therefore all courses that a user either has joined or created will be displayed here. To not take up too much space each course is a collapsed panel that can be opened to show any gamification elements available in the course. By clicking on the name of the course the user will be redirected to the page for that course.

It is also possible to join a course by selecting a course from a list. This list is displayed in a modal opened by the button above the courses you have joined.  The list does not only show the course code or the name of the course but also the creator of the course. This is because there can be many courses with the same name and/or code.

On the right half of the page you can see which courses you have requested to join and can the requests be canceled. The courses that have allowed students to join the course directly without any confirmation from the teacher will of course not be displayed here. Below the requests there's a list showing any invites sent by teachers to the user. The invites can be accepted or declined.

\subsection{Course}
An user can be a student or a teacher of a course, so it fell naturally to have two different course pages since both pages have their own functionality and usages. One view is for students and the other one is for teachers of a course. An user can also create courses which it will be a teacher of. 
\subsubsection{Student course page}
The student course page can be viewed by students that are registered on a course. To have a good overview panels, similar to the user page, are used to display different gamification elements that may be available to the course and assignments which is sorted by groups. The assignments are displayed in collapsible panel groups that allows the student to view the assignments and groups in an efficient way. If an assignment is clicked the student will be directed to the assignment page. 

% I might have described the gamification elements a bit too much here?
Depending on which gamification elements that are enabled for the course students will have different layouts. If for example the adventure map module is enabled a student can see its progress in the map, if it is disabled the student will not see it. Anyway, let us say all gamification elements are available, then the student would first see a progress bar. This displays how many assignments the student has completed of a total amount. Next is the badges, this shows all badges a student has managed to collect, depending on which test and assignments it has passed. After that comes the leader board. The leader board displays the score for each student in sorted order. Finally the adventure map, it presents groups and assignments in a way so the student can see its advancement and what is left. 

\subsubsection{Teacher course page}
The teacher page has a bit more functionality than the student page. It is here teachers can view and modify their courses and create groups, badges or manage adventure map. A teacher can see the course's description and click on edit to update the course. Under the title ``Assignments'' the teacher can view all assignments for the course, or create a new one. It is also possible for the teacher to create groups by a simple button click. Adding assignment to a group is done using an intuitive drag and drop method. 

If the adventure map is available for the course there is the possibility for the teacher to arrange assignments by groups and place them out on the map. If badges are an enabled feature a teacher can create a new badge, associate it with assignments and tests that is necessary for a student to pass before it can receive the badge.

For a comprehensive layout all student related information are put in a column on the right side. Here the teacher can see requests from students, invitations the teacher have sent to students and also students that are enrolled to the course. There is also options for the teacher to accept or decline a request from students, delete an invitation and add a student to the course. 

Adding a student to a course can be done in several ways. A teacher can either search for a student using its username and then send an invitation, or the teacher can generate an invitation link. The invitation link can be sent out to students who can then use it to join a course directly. It is also possible for the teacher to view invitations links, to see its code, how many times it has been used and when it expires. There is also the possibility to cancel the link.

The teacher can also toggle between student and teacher view. The teacher can use this to go through their own courses with groups and assignments to control so that everything is in order. 

\subsubsection{Create course}
Creating a course is possible for both ordinary students  and LTU teachers, however students are only allowed to create a maximum of three courses, LTU teachers can create any amount. The create course component is also used for updating a course. A course consists of a name, course code, description and the alternatives enabled features, public and auto join which is visualized as check boxes.

Enabled features are made up of progress bar, badges, adventure map and leader board, this decides if a gamification elements should be available in the course or not. Public means that the teacher will make the course public so students can see, request access or join the course. Auto join means that students will be able to join the course freely so no request is required. The course description uses markdown, same as the create assignment page. 

Then a course is submitted, an API call to backend to create a new course will be made. If it was successful the response will be used in course service to add the new course and with subscription all concerned components will be notified. 

\subsection{Assignment}
\subsubsection{Assignment Page}
The assignment component is used for displaying and for submitting a solution to an assignment. The component consists of three key fields. A feedback field for displaying server response from tester. A description field for showing the assignment description and finally an in-browser editor. 

The decision to implement an in-browser editor for writing the code was made early on as it was seen as more intuitive than having to write your code using your own editor and then uploading files to the site.

First iteration of the assignment component implemented an editor called CodeMirror, however this editor was dropped and replaced by an implementation of ace editor, which had far many more options in terms of customization such as size of editor and theme. Ace editor also had better support for switching between syntax highlighting of different languages, while CodeMirror required us to reload editor upon switching language. 

\subsubsection{Create Assignment}
The create assignment page is used for both creating and editing assignments. An assignment consists of a name, description, available languages and flag whether or not the syntax of the code should be tested using lint. 

The assignment description implements markdown, which is a format language for parsing text to correct HTML. This allows a teacher to add different sized headers, lists, embedded pictures, links, code blocks with syntax highlighting. Early version of the create assignment page implemented a live-preview for writing the description, however got removed to maintain consistent design with the create course page. 

The create assignment page also contains a field for creating and editing input-output tests. A teacher can add multiple tests for verifying the code against multiple inputs.

Once an assignment is ready to be submitted, an API call to backend to create new assignment will be made and if that call is successful, additional calls for adding input-output tests can be made. 

\subsection{Services}
In AngularJS, a service is a function or object that is accessible from anywhere in the application. On the frontend, services are used regularly as a means to reuse code, such as API calls to backend, as well as to maintain data across several pages. In the following sections some of the more important and well-used services are detailed.

\subsubsection{BackendService}
The BackendService is the center of all our communication with the backend. Every component or service that needs to either get or post information to or from the backend does so through this service. By requiring all calls to go through this service, we can perform all error handling in one place as well as ensuring that all calls are sent in a similar fashion. All calls expects a response with a JSON body. If there is no relevent information to send in the body an empty JSON body should be sent.

All of BackendService's functions return a Promise that will eventually resolve to the response from the backend. In the event that a request fails, a Toast containing the error message is displayed to the user. 

\subsubsection{AssignmentService and CourseService}
The AssignmentService is used to keep track of assignments belonging to a specific course. Information about assignments are stored in the service to remove the need of fetching assignments all the time when navigating the course page (e.g., going from the course page to the assignment page and back). The service is also used to keep the information up-to-date with changes that comes as a result of user action. For example, if a teacher creates a new assignment, we only fetch the new assignment from backend and adds it to the already saved list of assignments.

The CourseService fulfil the same purposes as the AssignmentService, but for courses instead of assignments as the name suggests.


\section{Backend}
The backend of GPP was developed using javascript running with NodeJs as server framework and Express as a webframework to handle routing. NodeJs was chosen because it was the most established framework with wide support and multiple extensions, such as Express.
The document database MongoDB was chosen for the project instead of using a relational database, the reasoning for this decision was to have a database that was built with scalability in mind and dynamic schemas but also because of the simple reason that it is easy for the programmers to read and understand the data inside the database thanks to MongoDBs JSON-like document structure.\\
\\
A coding-standard was quickly decided on to make sure the code style was somewhat coherent effectively making it easier for different members to take over other members code if needed.\\
\\
As a way to keep the project effective it was determined that it was important to implement continuous integration as a part of the project to have a modern workflow where you could push changes to git and have them automatically built on a live test server. Using continuous integration was also a good way to separate production and development builds and a way to ensure that production builds always kept a certain standard and robustness. The framework that was used for continuous integration was Jenkins which is one of the most popular frameworks.\\
While the idea of how continuous integration was supposed to be used in the project was quite clear, the concept wasn't put to as good use as it could have. In the first half of the project, the builds generated from Jenkins wasn't really put to use. This was mainly due to communication errors and the fact that the builds weren't needed as much. For the second part of the project the builds were used more. The way that the deployment with Jenkins was setup was to build from two branches from GIT, one from the master which was supposed to build for production and another from the backend branch which was supposed to build for development. The idea was that the development build would always contain the latest changes in backend. As the project reached its deadline it became increasingly important to also keep the backend dev builds stable because of the fact that the frontend team were depending on the backend dev build when developing.
Because of the increasing demands on having stable backend dev builds, a new requirement was setup for the continuous integration where a new development dev build was only built if it passed a set of unit-tests for the application.\\
\\
Connected to the reasoning of building the backend in different environments/modes such as production or development, it was important to load different configurations depending on what environment the backend was supposed to run. For example, the production environment should not use the same database as the development environment. Because of this, it was important to be able to run the backend in different modes and dynamically load the correct configurations based on the mode that the backend was running in, hence a system that did this was implemented.\\
\\

%Needs to be more filled out with technical details by other members of the Backend Group

\section{Tester}
% intro, what is Tester trying to do?
% we want broad language support
Tester is a tool for testing code in a variety of programming languages in a safe and isolated environment. To fulfill the needs of the GPP, Tester needs to support a  broad varierty of languages, so that the range of courses that can benefit from the platform is not limited. Tester also strives for the functionality to test code based on different merits to support the creation of varied, fun and interesting assignments.

\subsection{Sandboxing}
%TODO ref to docker
An important feature of the Tester is that the code is tested in an isolated enviroment. The main reason behind doing this is that we are executing potentially malicious code. Allowing such code to be executed without isolation is bound to cause service disruptions or information leaks. 

To achieve isolation all code testing takes place in so called Docker containers equipped with the tools necessary to compile and run the code. A container is a virtualization (simulation) of an operating-system with a separate user-space. What this means is that programs running inside the container can only access the contents and devices that are assigned to the container. By running the tests in Docker containers, any malicious code cannot gain access to information it shouldn't have access to or cause any disruptions to the service, since such code will at most crash the container which has no long-lasting effects on the service offered by Tester.
\subsection{Manager}
% TODO: add discussion of different ways of managing load and what we chose
% (Start containers on demand? Keep a pool of containers ready? Predict load or just react? Allocate different amounts of resources for different languages?)
\subsection{Test types}
% unit tests are hard to write
% I/O tests are language-independent which is good if we want broad language support
When deciding which test types had the highest implementation priority, an important factor was the time expenditure required by the creator of the tests. If writing tests for a single assignment takes a considerable amount of time the platform will become too impractical to use. For this reason tests such as Unit Tests are not supported. It would be time-consuming for lab supervisors to write tests. %idk what i wrote here

Another deciding factor is the language-dependency of the unit tests. If a test requires language-specific implementation the extensibility of the language support suffers. This is one of the reasons why the I/O and Code size tests mentioned below are excellent choices, they are both language-independent and require no additional implementation for every language supported. Although Lint tests also mentioned below does not have the independency property, it has been implemented for a smaller number of languages as a proof of concept.
%Should we explain what each test type is? or are they obvious enough?
Currently, Tester supports three types of tests:
\begin{itemize}
\item Input/Output (I/O)
\item Code size
\item Lint
\end{itemize}


\subsection{Extensibility}
In order to achieve broad language support, it needs to be easy to add new languages. Two steps are necessary to add support for a new language:

\begin{itemize}
\item The manager has a makefile that produces docker images for each supported language. The makefile needs to be extended with a new target that installs language-specific dependencies.

\item The Node instance running in the container needs to have a module for each supported language. Each language module exports two functions: \texttt{prepare}, which produces a file that can be run with the \texttt{run} function. The \texttt{prepare} step is used for compiled languages, where a binary must be produced before running the program. Some languages, such as python, are run directly in the interpreter. By making preparation a separate step, the program does not need to be recompiled for each test.
\end{itemize}

\subsection{Language availability fallback}


 \chapter{Discussion}

\section{Evaluation}
Initially, we set out to build a system that would make learning programming more fun and less tedious. This would involve gamification as a mean of picking up weaker students that would otherwise fail and give up in the face of overwhelming challenges. In addition, we wanted to remove, or at least lessen, the requirement of having personnel hired to correct labs by using automated correction.

To solve the automation, a system was developed that would create sandboxes, perform simpler tests and allow easy extensions. This way, other maintainers could add their languages of choice to the solution at a later time and select which testers they would like to use based on what languages are available and what latency they provide. Furthermore, this system was designed to scale out horizontally, meaning that if the system needs to be used by a larger amount of people, more hardware can be added to match the increased load.

Something that was not planned, but came as a consequence of how the tester was developed, is that the tester is so generic that it can be used for other projects as well. This opens possibilities for authors of similar services and games to develop new frontends quickly that make use of the service instead of reinventing it.

On the other hand, an issue that teachers of more advanced courses may have with the system is the lack of real unit tests. Without unit tests, making conclusions about application internals is nearly impossible and requires student interaction. However, an attempt was made at implementing unit testing for \emph{any} language, which showed that it could be done, see section~\ref{sec:testtypes} and section~\ref{sec:unittests_future}, but was eventually thrown away as the implementation proved both messy and hard to set up for teachers.

For the other part of helping students discover the joy of programming, like we already have, we have built a website that is both a game and a learning platform. While the game itself is not the most exciting piece of work to see the light of day, it does not spoil students into requiring games to learn either. By keeping the process of creating new game elements simple, we hope that it will be possible for others to create new elements on their own. This includes the ability to create elements that do spoil users and other bad things, but that will need to be formally evaluated at some point.

The provided modules showcase some simpler mechanics that can be woven into the service and that expose both positive and negative traits. For instance, the progress bar \emph{may have} the negative effect of not promoting learning beyond the platform, but to prove that it does, empirical data must be gathered both from the platform and from real life. Our system lets educators and researchers gather this data by running courses with and without game elements included. This data may then be compared to the success/drop-out-rate of the students attending the course.

Unfortunately, an important mechanic that we had hoped to implement, but were unable to realize in time, was group effort. A common theme discovered in the prestudy was that letting students cooperate and interact with other students introduces some interesting boons. Some angles that could be explored are further documented in section~\ref{sec:social}, but were considered too big of a risk to actually commit to at the late stage of development when the single-player elements were finished, as it might carry the hidden requirement of editing the database schema.

Fittingly, this brings us to the issue of maintainability and extensibility in the database. It was discovered, when introducing the game elements, that the database structure that had been developed could not support the level of modularity we wanted to offer. As an effect, storing custom data for game elements may require both API modifications and updates to the database schema. For the amount of game elements included in our release, it serves fine, but in extension it might not scale.

In conclusion, the service is able to provide a fun and engaging experience for students that does not necessarily spoil them into not studying. Meanwhile, educators are able to create courses that allow students to complete assignments and get rewards appropriately. This is done in a manner that can partially replace lab assignments and retain more students, all the while saving working hours that would otherwise be spent on correcting labs. Lastly, new game elements can be created by educators and researchers alike to expose data that is critical to research within the field of gamification within education.

% TODO: connect with future work and A/B testing Even without A/B testing in the system, simpler statistical measurements should % TODO: mmention that not all elements are good% Also, with the modular approach taken to the game elements, new games can be created without too much work.





\section{Workflow}
There were 15 students in a project that was originally dimensioned for 7--9 students. Apart from the common issues that arise when managing very large groups, the oversized project group made members who aren't as used to voicing their opinions express themselves even less. This characteristic is also something that was mentioned in the conducted interviews and the solution is usually to either help people trust one another (teambuilding), or to remove the requirement of trust (anonymization). 

The project specification didn't specify all details regarding the requested product. This lead to a long prestudy wherein a lot of time was devoted to answering questions that maybe could have been partially answered by the project owner if asked more thoroughly. No technical work was initiated before the prestudy was done, primarily out of the fear of performing work that would be thrown away if the project goal changed. In effect, a lot of people did very little work during the four first weeks. In hindsight, a part of the group could have started looking into other technical parts that would be needed regardless of the results acquired in the prestudy.

Delegating all inter-group communication to the leaders was a challenge, probably because the needs were far too large. Since the group leaders were supposed to inform all their members about changes in the other groups, communication suffered. When this was discovered, an informal decision was made to encourage participants to visit the project room. We hoped that by doing so, communication between developers would become more active and clear, which it did. However, as it was an informal decision, it's possible that not all participants understood this shift. Something to take away from this is to make sure that all decisions are clear to everyone affected before committing to them.

The project leader, who should have an overview of the development, got information about the progress from the group leaders. This information was superficial and unable to compensate for the lack of communication with the developers. This lead to a lack of technical depth in the biweekly presentations. To increase the technical content of the presentations, the group leaders had to become more involved in them. The project leader should have been more present in the project room when the developers were there to gather information about the development progress. In conclusion, having too many levels of leadership did not work well for an organization of this size.

Structuring a functional workflow with 15 members was a big challenge and an invaluable experience for all members since no one had any previous experience in working with such a large group. Even though the group faced several issues and challenges, a product was developed that each and every one could be proud to have been part of.

\subsection{Project Planning}
The agile development style was well-suited to the biweekly presentations. However, the group was not completely satisfied with \taiga{} as a sprint planning tool. There were many features that were not used because they seemed to only increase the burden on the developers. The group leaders found that transitioning between sprints required a lot of manual work. Partly as a result of the problems with the tool, tasks gradually transitioned to being tracked on a whiteboard in the project-room. This lead to a clearer picture of the current tasks for on-site developers, but compounded the previously mentioned problems for remote developers.


\section{End-Product}
We set out to build a website that could help examiners improve their courses by using gamification. To do this, both student interaction and teacher management had to be considered to create a pleasant user-experience.

Most notably, we aimed to minimize the unnecessary feedback loop between students and lab-supervisors. By creating an extensible and scalable tool for automated testing of code, we could provide an environment that our platform---and other platforms---can use to test assignments.

By implementing a framework that would allow quick implementations of new elements, a platform for researching new gamification tools was born. After implementing a new element, it can be evaluated by looking at the completion statistics. However, there is much work that can be done on this part to empower teachers and gamification researchers with the data they need to make conclusions regarding their original research.

As a deployment note, the solution should be fairly easy to install for the IT department. Our plan of action for doing this was to deploy frontend and backend as docker containers together with a haproxy docker container for load-balancing. Configuring these would be done using configuration files which the person deploying the solution could modify before building the docker containers, and thus set the correct settings for their setup. However, the tester environment requires docker to run, and should probably not be nested\cite{nesteddocker}, so that needs to run on its own virtual machine. Some additional work may be needed for the first deployment to generate a working configuration.


%\section{Conclusions}
%Gamification has become a large part of motivation. By getting rewards for performed works the motivation and satisfaction increases. Gamification with immediate feedback in programming is fairly new since learning programming has grown and spread to lower ages, a tool for learning programming quickly is needed.  
The platform provided is dynamic in that way that there is no limitation in the difficulty in the assignments, the creator of the assignment is that one which sets the level. By that said, even since the target is university students, the platform is also applicable to younger students. There are endless choices for new gamification parts that can be implemented, only the imagination puts a stop to which parts can be implemented. The platform delivered is an initial working product that is usable but can be expanded to fairly low cost. 


 \chapter{Future Work}

\section{File Upload}
A feature that is not currently supported is the option for users to upload files to the platform. This limits the teachers' abilities to upload backgrounds for the map and requires all students to write their code using the site's in-browser editor. The Ace-editor we are currently using is limited both in terms of features and customizability. Further, this would allow the student to write and save their code offline in the case that either the site is down or if they are in a location with no network connection.

Implementing this feature could require support on both the frontend and backend. If the file is stored as a file in the database, support for storing files would be required. But the code could also be stored as a string similar to how code submissions through the in-browser editor are stored, which would only require some parsing on the frontend side.


\section{Statistics}
\begin{wrapfigure}[16]{r}{7cm}
    \centering
    \includegraphics[width=\linewidth]{img_src/progress_over_time.png}
    \caption{What a time-progress graph could look like. Here, each color represents an assignment, but it could also be split into individual graphs.}\label{fig:progovertime}
\end{wrapfigure}
As of right now, there is a simple statistics page that is implemented on the frontend for showing statistics for a specific course. A teacher can see pie-charts for each assignment in the course and how many students that has completed that assignment of the ones who are registered to the course.

The current implementation is using the API call to get all features for that specific course. This API call returns a lot of data that is unnecessary, and in some cases might not be correct to send to frontend. The data used for statistics has to be individually picked out from the massive response that the API call returns. In the future, it might be good to create a route from backend that returns the amount of students that has passed a specific assignment.

In the future, it might be nice to add a graph like the one in figure~\ref{fig:progovertime} for each assignment plotting when students pass a specific assignment. This feature would require the database to keep track of which dates students first pass specific assignments.


\section{A/B testing}
A/B testing can be applied in various ways to measure how effective improvements and updates are to a service. This essentially means utilizing 2 different groups of users to discover behavioural patterns and interpreting the result from the difference in performance. In GPP this might be used to improve the user interface by finding better color combinations and/or element positions. More importantly, tests could be run by teachers and researchers to determine which elements work and which don't.

One way of implementing this in GPP could be to divide a fraction of the user base into groups, expose them to changes and monitor its effects. For instance, testing the adventure map might show a significant increase in performance for group A whilst group B shows no change. Furthermore, change could possibly be seen in the second group---despite there being no change to the group itself---as the total amount of completed assignments increases in both groups.

Given that enough users participate in the A/B study, it can be shown with statistical significance that a gamification element is either profitable or inconvenient. In extension, this allows examiners to convince their faculties by showing them data proving its usefulness, allowing them to implement it into their courses.

The reason for this not yet being implemented is primarily that it requires the correct statistics to be available for extraction. Unfortunately, little work was devoted into statistics aggregation on the backend, meaning that until recently, implementing statistics-based features was very cumbersome. With a further developed statistics feature, implementing A/B testing should be feasible.

\subsubsection{A Twist}
\begin{wrapfigure}[14]{r}{6cm}
    \centering
    \includegraphics[width=.6\linewidth]{ab-loadbalanced.png}
    \caption{An alternative approach to A/B testing using the load-balancer.}
\end{wrapfigure}
Depending on what kind of statistics the A/B tests should determine, an alternative, more general approach could be taken. This works by implementing the testing \textit{on top of} the solution, instead of within, and may be easier to get started with. The drawback is that all tests would be done on a more coarse granularity.

Rudy Lee provides an example\cite{rudylee} for how this can be setup using the recommended HAProxy load-balancer. By first letting a balancing algorithm (round-robin, random or minimal-load) send users to a specific site and then setting a cookie to track and direct users on repeated visits, statistics like retention rate can be determined when doing site-wide changes. Unlike an A/B tester implemented inside the platform, this tool cannot capture changes done to single courses, only global ones, which limits the usefulness to educators.


  \appendix

\chapter{Workshop}
% TODO: this should be place somewhere more appropriate
\chapter{Workshop}
\label{sec:workshop}
As a part of the prestudy, educators and a few students were invited to a platform-development workshop. During the event, participants were divided into groups and asked to develop their own mockups for educational services according to the script below.

\section*{Introduction (10min)}
We are studying the fifth year at computer science program at LTU and for now doing our project. As a part of our prestudy are we hosting this workshop where we will investigate possibilities through a couple of scenarios. You will do a couple of assignments in groups and then present your result and discuss these with others.

\section*{Scenario 1 (10 min + 10 min discussion)}
You have just started a course in which you will learn to program in a language that you don't find interesting. The teacher has given you a lot of recommended exercises but they don't seem very interesting. Which type of system or moments would motivate you to do the assignments. Associate freely.
\begin{itemize}
\item Which subject do you think has this problem?
\item Do the assignments need to be presented? If so, how?
\item Which material is needed?
\item If you were to make a list of highlights, what would be on it?
\end{itemize} 

\section*{Scenario 2 (10 min + 10 min discussion)}
After many years, you find yourself teaching that boring course. The boss comes in to your office late one Friday afternoon and says that the course moments are too few and that you need to update them before Monday. You sit down as the motivated teacher you are and contemplate what kind of system that would motivate students. Suddenly, you remember your old concept.

You want to make this work in the long run, which limitations must you as a teacher do?
\begin{itemize}
\item What is most time consuming?
\item Is it possible to sort your highlights by time required?
\item How many hours would be required to develop and maintain the solution over time?
\end{itemize} 

\section*{Scenario 3 15 min + 10 min discussion}
Now your system works well for that boring course and with a little tweaking, it should work with all those new courses you'll be giving. However, it is 2017 after all and the students are crying for a web-based tool. What does this tool look like? Model a web-page to fit your tool.

\begin{itemize}
\item How do you handle inputs?
\item What do your highlights look like in practice?
\item Do you see any limitations?
\end{itemize} 


\section*{Results}
A set of project proposals for a web-development course was written as a result of the workshop, these proposals can be found at \url{https://github.com/ax-rwnd/m7011e-projects}. Furthermore, a list of ideas and reflections noted in the workshop

\subsection*{Scenario 1}
\begin{itemize}
\item Let assignments interlink.
\item Be able to choose which assignments should be done.
    \begin{itemize}
    \item Hard to grade, every goal of the course need to be met.
    \item Assignments with similar content.
    \end{itemize}
\item Constructive alignment, how goals, assignments and examination interacts.
\item Early and quick feedback.
\item Formative assessment, continuous feedback.
\item Group assignments, hide the personal result.
\item AI Avatar that gives feedback.
\item Lot of visualization of the result.
\item Present the purpose of the assignment.
\item Different angle of the design depending on the user which is solving the assignment.
\end{itemize}

\subsection*{Scenario 2}
\begin{itemize}
\item Not everything in a course should be changed at the same time.
\begin{itemize}
\item Old assignments could be divided.
\item Shorter feedback loop.
\item Old assignments with new context.
\end{itemize}
\item Updating the course introduction is cheap and makes the goals of the course more clear when the knowledge is to be tested.
\begin{itemize}
\item Thoroughly, min/max, median or average.
\item In the end there is a written exam.
\end{itemize}
\item Retain results from previous students
\begin{itemize}
\item Keep old posts in forums.
\end{itemize}
\item Peer-to-peer feedback.
\item Test an implementation even if it's bad to receive feedback.
\item Students create assignments for each other.
\end{itemize}

\newpage
\subsection*{Scenario 3}
\begin{itemize}
\item Anonymous questions
\begin{itemize}
\item ``Did you understand what I said?''
\item Direct feedback.
\end{itemize}
\item Direct chat between students and teacher.
\item Auto generated feedback to the teacher.
\begin{itemize}
\item Aggregate data from relevant measurements.
\end{itemize}
\item A student knowledge bank.
\begin{itemize}
\item Wiki format.
\end{itemize}
\item Code correction / evaluation.
\end{itemize}

\vfill
\subsection*{Photos}
\begin{figure}[hb]
\centering
    \begin{subfigure}{.32\textwidth}
        \centering
        \includegraphics[width=\textwidth, angle=270, origin=c]{img/workshop1_resized.jpg}
        \label{fig:workshop1}
    \end{subfigure}
    \begin{subfigure}{.32\textwidth}
        \centering
        \includegraphics[width=\textwidth, angle=270, origin=c]{img/workshop2_resized.jpg}
        \label{fig:workshop2}
    \end{subfigure}
    \begin{subfigure}{.32\textwidth}
        \centering
        \includegraphics[width=\textwidth, angle=270, origin=c]{img/workshop3_resized.jpg}
        \label{fig:workshop3}
    \end{subfigure}
    \caption{Workshop}
\end{figure}
\vfill



\chapter{Installation Instructions}
The Gamified Programming Platform was built and tested on Debian GNU/Linux 9 (stretch) with \nodejs{} 8.6.0 and npm 5.3.0.

\section{Tester}
Tester consists of two components; Manager and Runner. Manager replies to requests from the backend and manages docker containers that run arbitrary code. Containers are used to ensure that some test $A$ does not interfere with some later test $B$ by modifying the execution environment.\\
\begin{enumerate}
    \item Clone the repo: \texttt{git clone \url{https://github.com/ax-rwnd/d7017e-project}}
    \item Change directory to the Manager folder: \texttt{cd d7017e-project/tester}
    \item Install the dependencies for the Manager: \texttt{npm i}
    \item (Optional) Select languages by adding/removing dependencies in \texttt{Makefile}. For instance the line \texttt{all: python27 python3 java c \# haskell} selects the languages Python 2.7, Python 3, Java and C, but not Haskell (since it's commented out).
    \item Run the Makefile: \texttt{make}
    \item (Optional) Set preferences for Runner in \texttt{config/default.js}. There, things like queue lengths and ports may be configured.
    \item Move back up to manager: \texttt{cd ..}
    \item Start Manager: \texttt{node server.js \{PORT\}}
\end{enumerate}

\section{Backend}
Backend is the state-managing component. It uses MongoDB to store information that it receives while processesing frontend requests and tester results.
\begin{enumerate}
\item Install och configure MongoDB.
\item Clone the repo: \texttt{git clone \url{https://github.com/ax-rwnd/d7017e-project}}
\item Change directory to backend: \texttt{cd d7017e-project/Backend}
\item Install dependencies: \texttt{npm i}
\item Configure database address/port in \texttt{Backend/config/default} and \\
\texttt{Backend/config/production}. IP/Port may differ between the files, should you want to use different databases for testing and production. To select one of these files, set the \texttt{NODE\_ENV} environment variable to \texttt{production} or \texttt{development}.
\item Start the backend daemon: \texttt{npm start}.
\item (Optional) Start backend in the foreground: \texttt{node ./bin/www}
\end{enumerate}

\section{Frontend}
Frontend is the part that the users see. It builds on Angular for UI and contacts backend for functionality.

\begin{enumerate}
    \item Clone the repo: \texttt{git clone \url{https://github.com/ax-rwnd/d7017e-project}}
    \item Change directory to frontend: \texttt{cd d7017e-project/frontend}
    \item Redirect frontend to backend: \texttt{sed -i "s/ \textbackslash (backend\_ip: \textbackslash )'.*'/ \\ \textbackslash 1'\url{https://{your\_backend}}'/" src/environments/environment.prod.ts}
    \item Tell were the global ip for frontend is: \texttt{ed -i "s/\textbackslash (frontend\_ip: \\ \textbackslash)'.*'/\textbackslash 1'\url{https://{your\_frontend}}'/" src/environments/environment.prod.ts}
    \item (Optional) Repeat step 2 and 3 for \texttt{src/environments/environment.ts}
    \item Move or link your ssl-certificates \texttt{ln -s encryption/private.key.default \\
    encryption/private.key \&\& \\
    ln -s encryption/server.crt.default encryption/server.crt}
    \item Start server: \texttt{ng serve --ssl 1 --ssl-cert ./encryption/server.crt \\
    --ssl-key ./encryption/private.key --live-reload false}
\end{enumerate}


\chapter{Interviews}
The questions asked during the interviews were as following:

\begin{itemize}
 \item Do you find that student appreciate e-tools?
 \begin{itemize}
 \item Which tools do you think is needed to engage students?
 \end{itemize}
 
 \item A large problem (at least at the computer science program) is that many students are shy and don't dare to ask questions. Do you think that an e-tool in the classroom could simply this, e.g.\ anonymous feedback services? 
 
 \item Do you experience that there are different types of students that respond on different types of motivation?
 \begin{itemize}
 \item If so, which trends have you experienced?
 \item Do you think that it is possible to reach more students by e-tools?
 \end{itemize}
 
 \item Are you familiar with gamification?
 \begin{itemize}
 \item How did you get in touch with it?
 \item Which pros and cons do you see?
 \item Which pitfalls have you experienced?
 \end{itemize}
 
 \item How much time would you as a teacher be able to spend to create assignments?
 \begin{itemize}
 \item How much time do you spent on creating regular assignments?
 \end{itemize}
 
 \item What do you need available to see that this tool will work?
 \begin{itemize}
 \item Which difficulties do you see by using a e-tool, e.g.\ \techio()?
 \end{itemize}
\item How can we motivate you as a teacher?
\item What would you think about if you should create a similar tool?

\end{itemize}

A summary of the results that arose from the the interviews:
\begin{itemize}
 \item Mandatory assignments would increase the participation frequency, but would be less fun.
 \item Public scoreboards may be demotivating.
 \item The results in a course be represented by a score.
 \item Gamified tools need to balance school and fun, or students will never get motivated to study by themselves.
 \item Levels/different difficulties can be used to avoid long feedback-loops.
 \item Even the weaker students should have a chance to solve assignments on the easier levels.
 \item Points and badges should be used with care, tools that abuse external motivation like this make students less responsive to internal motivation.
 \item Anonymous questions can be used for great good.
% \item Good for the lower grades. % What?
 \item E-tools need to be easy to adapt into courses, or they won't be used.
\end{itemize}


\chapter{GPP in Pictures}
\subsection{GPP in pictures}

\begin{figure}[H]
\centering
\includegraphics[width=0.8\textwidth]{img/gppinpictures/login.png}
\caption{Login}
\label{fig:login}
\end{figure}

\begin{figure}[H]
\centering
\includegraphics[width=0.8\textwidth]{img/gppinpictures/user.png}
\caption{User page once logged in}
\label{fig:user}
\end{figure}

\begin{figure}[H]
\centering
\includegraphics[width=0.8\textwidth]{img/gppinpictures/teacher.png}
\caption{Teacher view for a course}
\label{fig:teacher}
\end{figure}

\begin{figure}[H]
\centering
\includegraphics[width=0.8\textwidth]{img/gppinpictures/studentview.png}
\caption{Student view for a course}
\label{fig:student}
\end{figure}

\begin{figure}[H]
\centering
\includegraphics[width=0.8\textwidth]{img/gppinpictures/assignment.png}
\caption{Assignment page}
\label{fig:assignment}
\end{figure}

\begin{figure}[H]
\centering
\includegraphics[width=0.8\textwidth]{img/gppinpictures/adventuremap.png}
\caption{Adventure map with assignments}
\label{fig:adventuremap}
\end{figure}

\begin{figure}[H]
\centering
\includegraphics[width=0.8\textwidth]{img/gppinpictures/badges.png}
\caption{Course badges}
\label{fig:login}
\end{figure}

\begin{figure}[H]
\centering
\includegraphics[width=0.8\textwidth]{img/gppinpictures/editcourse.png}
\caption{Edit/create course page}
\label{fig:editcourse}
\end{figure}

\begin{figure}[H]
\centering
\includegraphics[width=0.8\textwidth]{img/gppinpictures/editassignment.png}
\caption{Edit/create assignment page}
\label{fig:editassignment}
\end{figure}

\begin{figure}[H]
\centering
\includegraphics[width=0.8\textwidth]{img/gppinpictures/editassignmenttests.png}
\caption{Tests and language selection for edit/create assignment}
\label{fig:editassignment2}
\end{figure}

\begin{figure}[H]
\centering
\includegraphics[width=0.8\textwidth]{img/gppinpictures/statistics.png}
\caption{Course statistics page}
\label{fig:statistics}
\end{figure}


\chapter{Promotional video for GPP}
A promotional video was created to market the project. The video needed to be approximately two minutes and the purpose of it was to show off the features of the platform.
To capture the audience attention while still describing the platform in under two minutes proved to be a challenge. A storyboard was constructed to plan and make sure the video would have clear purpose and goal.
\begin{figure}[h]
    \includegraphics[scale=0.2]{storboard-promo.jpg}
    \caption{Image of the storyboard for the promotional video}
\end{figure}
After the storyboard was completed, a lot of time was spent finding the right type of royalty-free music. Once a suitable song was found, the video recording of the system was done. Pictures of every page of the site was also snapped. However, once most of the site was captured, some visual changes were made on the site which resulted in the old recordings and pictures became unuseful, which meant that a second attempt at recording video and snapping pictures of the site was done.

After all preparations were done, the promotional video was animated, edited and rendered using Adobe After Effects. Due to the limited time of the promotional video and some creative decisions, the final product deviated somewhat from the original storyboard. The final promotional video can be found through this url: \url{https://www.youtube.com/watch?v=NiJk_54Qsck}



\addtocontents{toc}{\protect\end{multicols}}

\end{document}
