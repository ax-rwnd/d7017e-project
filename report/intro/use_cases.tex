The tool should be lightweight and easy to use. Some of the answers that was recieved from the interviews with teachers at Luleå University of Technology stated that tools tend to not be used if they are too complex and time consuming. A teachers time is limited, and while initial setup might take some time, the overall time to maintain the system should be low. The same applies for students using the tool. The process of registration and joining a course should be straight forward and fast.

The tool should be fast and responsive, and provide feedback back to the user whether or not the code pass the tests and requirements stated by the teacher. If the code would fail tests or simply not compile, feedback and hits to what went wrong should be provided. 

Gamification contains quite a few different type of elements, some which might not suit a specific course. As such it's important that the teacher should have the ability to choose which type elements to include.

The system should have two different types of roles, teacher and student.

\subsection*{Teacher}
A teacher logs into the website and creates a new course. In the create course form, the teacher has to define the name of the course, write a detailed description of the course and choose from a list of gamification elements to use in the course. 

With the new course created, the teacher wants to create new assignments for the course. On the create assignment page, the teacher designs the assignment by writing a detailed description of the task. The teacher then adds tests for the assignment from which the student's solution should be tested against. To get students to join the course, the teacher can send out invites.

\subsection*{Student}
A student logs into the website and joins a course. On the course page, the student can see all available assignments as well as the enabled gamification elements. The student picks an assignment and start working on a solution. Once the solution is done, the user submits the code for testing.

Within a few seconds, the website responds whether or not the students solution has passed all the tests. If the solution is correct, the students progress in the course is updated. If the solution is incorrect on the other hand, the page will display what went wrong.