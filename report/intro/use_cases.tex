For the tool to be useful, it should be lightweight and easy to use. Some of the answers that was recieved from the interviews with teachers at Luleå University of Technology stated that tools tend to not be used if they are too complex and time consuming. A teachers time is limited, and while initial setup might take some time, the overall time to maintain the system should be low. The same applies for students using the tool. The process of registration and joining a course should be straight forward and fast.

The tool should also be fast and responsive, and provide feedback back to the user whether or not the code pass the tests and requirements stated by the teacher. If the code would fail tests or simply not compile, feedback and hints to what went wrong should be provided.

Since the prestudy showed that there are both good and bad ways to use different types of gamification elements, this tool could be a perfect opportunity to test different type of elements. The statistics generated from running a course with these elements could be used in further research about gamificiaction. As some gamification elements might not suit a specific course, it is important that the teacher should have the ability to choose which type of elements to include. 

As an auxiliary tool for students and teachers the system is not meant to replace any existing grading in the course. This since it's very hard to test a student's understanding and how the student has solved an assignment. 

The system should have two different types of roles, teacher and student.

\subsection{Teacher}
A teacher logs into the website and creates a new course. In the create course form, the teacher has to define the name of the course, write a detailed description of the course and choose from a list of gamification elements to use in the course. Teachers might want to spend different amount of time setting up their course. Therefore some elements work automatically while others don't but allow for more customization so the teacher can make the course more interesting.

With the new course created, the teacher wants to create new assignments for the course. On the create assignment page, the teacher designs the assignment by writing a detailed description of the task. The teacher then adds tests for the assignment from which the student's solution should be tested against. A course with a lot of assignments might be overwhelming to students. To counteract that the assignments could be sorted into groups. This helps students divide the course into smaller parts and set goals that feels more attainable.

To get students to join the course, the teacher can send out invites to other users. To make it more simple and to also be able to invite people that have not yet logged in to the site invite links can be used. If the teacher would want to allow anyone to join the course it should be possible to make it open to everyone.


\subsection{Student}
A student logs into the website and joins a course, either by accepting an invite or finding one that they would like to join. On the course page, the student can see all available assignments as well as the enabled gamification elements. The student picks an assignment and start working on a solution. Once the solution is done, the user submits the code for testing.

Within a few seconds, the website responds whether or not the students solution has passed all the tests. If the solution is correct, the student's progress in the course is updated. If the solution is incorrect on the other hand, the page will display what went wrong. To not discourage students from trying hard assignments, there are no penalties from failing. Furthermore a student could try the same assignment multiple times.