In the system there should be two different roles, teacher and student.
The teacher is able to create courses, assignments and invite students to the courses.
The teacher will create a course by fill out the description and the name of the course. Then it's possible to choose which gamification parts that should be included in the course. The visibility of the course is chosen by two options public and auto join. When public, the students can see that the course exist and if auto join is chosen the students can freely join the course otherwise they have to request access. When a teacher creates a assignment the teacher fills out the name of the assignment and chooses between LINT-test and input/output test that should be applied on the assignment. The teacher can then write the assignment in with markdown for better structure and easier reading. It is now possible to create badges where the teacher chooses which badges should be applied to an assignment.\\

A student can be invited to a course or join courses. The student that want to solve an assignment can choose the assignment from a list of assignments on the course page or choose the assignment in the adventuremap if it has been selected by the teacher. The student can see the progress of solved assignment and rewards in form of badges, leader board and progressbar. When solving an assignment the student writes the solution and then submits it. The student will get direct feedback with a reward if the solution is correct otherwise the feedback will be a text showing which tests has failed.
