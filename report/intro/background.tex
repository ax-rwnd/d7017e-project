At \LTU\ there are many courses where programming is a major subject. Students sometimes face difficulties in knowing whether they have programmed an assignment correct or not until a lab supervisor or examine checks the code. This dependency on human interaction in general lead to slow feedback. Thoughts, reflections and memories regarding the assignments are often forgotten due to the time passed between writing the code and receiving feedback on the code.

Apart from this, students frequently find that their studies in programming are useless because they do not undestand where- and how this knowledge can be applied. In other words, they lack the context they would need to actually motivate themselves to learn programming. By providing context, the students can have a better understanding of how they can apply their knowledge until they actually encounter the real-life uses.\\
\\
To shorten the time required for feedback and make assignments more fun, an interactive web tool for coding practice was requested. There are many different applications on the internet that already does various parts of this, but none that are freely available and suitable for usage in connection with university courses.

None of the tools that were investigated were able to fill all the functionallity that teachers may want to use to create exercises. The tools were either build like games or had very little- to no pedagogical parts and the few pedagogical parts were quite boring. It also turned out that many of the tools were too complicated for teachers to implement in existing courses. The tools would need to be simplified to be useful or too much time would be spent on maintenance.\\
\\
A way to make assignments more fun was to apply gamification to the requested web tool. The purpose of gamification is to take elements from games and apply them in other areas to create an increased commitment within the applied area. The target group for the platform was mainly students of Computer Science and Electrical and Space Engineering with a programming direction.\\
Other educations within LTU that could contain programming parts were also interesting and kept in mind, for example education for teachers where programming will become a central part in the future because of a revised curriculum that passed in the fall of 2017 for elementary school and high school.\\
\\
To reach the goal of the project - which was to develop an interactive web tool for coding practice that shortened the feedback time and increased the fun of making assignments, a platform that embodied these ideas was needed. The goal was to have a platform that was easy for both users and teachers alike. Gamification was a big part of the platform, but it became apparent that not all teachers wanted to use these elements and because of this it was important that the gamification features were optional and that it was easy to enable and disable the features for each course as the teacher saw fit. Because of this, a big part of the goal during development of the platform was to make features modular and simple.\\
\\
The expected result of the project was closely connected with the goal of the project. The platform needed to include gamification elements that stimulate and inspire the  students so that they would want to program more and become better at programming. Further it was important that students should be able to write the code inside their web browser and be able to instantly verify if they have passed the requirements for the assignment or not, effectively shortening the feedback time.\\
\\
To further motivate the students and make them feel that programming is fun, the platform needed to support progress and different levels of difficulty so that every student can feel properly challenged. In the more classic educational model, practical programming tasks would usually only amount to a few assignments with large increases in difficulty, where some students would feel overwhelmed by the difficulty of the assignments and other students would be underwhelmed by the reward provided by solving said assignments
By offering smaller increments in difficulty, it would be easier to keep students from getting stuck at specific tasks, even if the task is considerably harder than the first task the student did. With different difficulties, every student could feel that they received challenges and rewards that correctly represented their level of knowledge.
Finally the platform shouldn't be limited to LTU even though LTU students are the primary users of the platform.