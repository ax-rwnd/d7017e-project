At \LTU\ (LTU) there are many courses where programming takes a significant role. Students sometimes face difficulties in knowing whether they have solved an assignment correctly until a lab supervisor checks the code. This dependency on human interaction in general leads to a long feedback loop, as well as taking up too much of the lab supervisors time which could instead be used to help more students. 

The feedback loop can be drastically shortened with the help of automated testing. Such a tool could lead to less load on the lab supervisors, potential cost savings and more efficient time usage for students and teachers.

On top of this automated testing a web application could be built. Since games are a common way of motivating people an idea is to introduce gamified elements to make learning more fun. From a teacher's point of view this could increase popularity and participation in courses.

The main problem introducing such a tool in education is that it has to be easy to use. This applies to all users, teachers and student alike. If the system is too complicated to use, the time saving aspects of using automated testing might be lost. At the same time it should provide things not already present in traditional education. Furthermore the same tool needs to be adjustable for different types of courses and assignments to fit their needs. By having the system offer teachers a selection of features they can decide what they want to use in which course.