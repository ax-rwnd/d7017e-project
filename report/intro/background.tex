At \LTU\ there are many courses where programming is a major subject. Students sometimes face difficulties in knowing whether they have programmed an assignment correct or not until a lab supervisor or examine checks the code. This dependency on human interaction in general lead to slow feedback. Thoughts, reflections and memories regarding the assignments are often forgotten due to the time passed between writing the code and receiving feedback on the code.

Apart from this, students frequently find that their studies in programming are useless because they do not undestand where- and how this knowledge can be applied. In other words, they lack the context they would need to actually motivate themselves to learn programming. By providing context, the students can have a better understanding of how they can apply their knowledge until they actually encounter the real-life uses.\\
\\
To shorten the time required for feedback and make assignments more fun, an interactive web tool for coding practice was requested. There are many different applications on the internet that already does various parts of this, but none that are freely available and suitable for usage in connection with university courses.

None of the tools that were investigated were able to fill all the functionallity that teachers may want to use to create exercises. The tools were either build like games or had very little- to no pedagogical parts and the few pedagogical parts were quite boring. It also turned out that many of the tools were too complicated for teachers to implement in existing courses. The tools would need to be simplified to be useful or too much time would be spent on maintenance.

