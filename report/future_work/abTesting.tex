To increase and maximize an outcome of interest A/B testing could be performed.
This essentially means utilizing 2 different groups of users to discover behavioural patterns and
interpret the result to improve affected areas.
This could also be implemented in an automated way that involves closly monitoring users to interpret their behaviour.

By A/B testing GPP you could for instance improve the user interface by finding better color
combinations and/or element positions. It could help finding a design pattern for course structures that aids both students and
teachers usability, scalability and learnability. As well as help finding out what gamification elements are the most/least
popular and possible areas of improvement.

One way of implementing this in GPP could be to divide a fraction of the user base into groups, expose them to changes and monitor its effects.
For instance testing the adventure map might show a significant increase in usage in group A whilst group B shows no change.
You could also see that despite no change in usage in the second group the total amount of completed assignments increases in both groups.
Thus the evalution follows that implementing the adventure map has been shown a benefitical thing to do.
After evaluating you have the option to gradually expose more users until its fully implemented or revert the changes.

This could potentially lay the ground for financial investments, furthering the research or time investment in
learning/developing GPP.
