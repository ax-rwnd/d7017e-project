\chapter{Introduction}
\section{Background} 
At \LTU, there are many courses wherein programming is a major subject. Students sometimes face difficulties knowing whether or not they have programmed correctly until a lab supervisor or examiner checks the code. This dependency on human interaction generally leads to slow feedback, and reflections regarding the programming tasks are often forgotten between the time of writing and the actual checking.

Apart from that, students frequently find that their studies in programming are useless, because they do not understand where- and how it can be used. In other words, they lack the context they would need to actually motivate themselves to learn programming. By providing a context, the students can be `fooled' into thinking they have a use for their knowledge until they actually see the real-life uses.

To improve the time required for feedback and make the assignments more fun, an interactive on-line tool for coding practice was requested. There are many different applications on the internet that already does parts of this, but none that are freely available and suitable for holding university courses in.

None of the tools that were investigated were able fill all of the niches that teachers may want to use to create exercises. They either were built like a game and had no pedagogic parts or provided pedagogy only and were quite boring. Also, many of the tools investigated were too complicated for teachers to implement into existing courses, so they would need to be simplified before taken into use, or too much time would go into maintenance.% , should  the platform otherwise it will not be used. The bottleneck concluded is the teachers mentioned above, they need to easily and in a few hours be able to implement a course in the platform. To fill this absence of information, interviews were conducted with teachers, students and finally a workshop was setup with teachers to try to broaden the picture of what is missing and what they need.

% NOTE: giving students challenges is very much not the same as students being challenged
%The idea is that with direct feedback and progress visualizations, students may find that programming has actual value and use. Partly since they are no longer limited to the strict confines of lab sessions to get their work done, but also because they get to feel that their work has both context and some `real' meaning.

%TODO: move this somewhere appropriate!
%Furthermore, by offering smaller increments in difficulty it's easier to keep students from getting stuck at specific tasks, even if the task is considerably harder than than the first one they did in some series of tasks. With different difficulties, every student can feel that they receive challenges- and rewards that correctly represent their level of knowledge. In the classical model, practical programming tasks would usually amount to a few assignments with steeply increasing levels of difficulty, where some students would feel overwhelmed by the difficulty of the assignments and other students would be underwhelmed by the reward provided by solving said assignments.




\section{Goals}   
% Uh, was it?
%The goal of the project was to find a teaching method for the students that facilitates learning for both basic and advanced programming by using gamification theories.

The purpose of gamification is to take elements from games and apply them in other areas to create an increased commitment within the applied area.
The target group is mainly students studying at Department of Computer Science, Electrical and Space Engineering  with a programming direction but also other educations within LTU that contain programming parts, for example teacher education where programming will become a central part in the future because of a revised curriculum for elementary school and high school that passed in the fall of 2017.

To reach the goal of the project, a platform that embodied these ideas was needed.
The goal was to have a platform that was easy for both users and teachers alike. Gamification was a big part of the platform but a realization that not all teachers wanted to use these elements was reached and because of this it needed to be simple to activate and disable gamification features for courses as the teacher saw fit. As such, a big goal during the development of the platform was to make features very modular.

\section{Expected result / requirements}  
The expected result was to deliver a web-based platform for learning programming. The platform should include gamification methods that stimulates the student so that they want to program more and be better at programming. The students should be able to write their code in the browser and then be able to validate if they have the correct outcome of the assignment. The platform should support progress and different levels so that every student is challenged. The platform shouldn't be limited to LTU even though LTU is the primary source that will use the platform. 

\subsection{Initial requirements} 
 \begin{itemize}
\item Full solution for writing and testing code.
\item Immediate feedback.
\item Show progress/statistics.
\item Involve gamification.
\item Support multiple programming language.
\item Teachers should be able to construct assignments.
\item Easy navigation.
\item Log in by CAS. 
\item Primary target is LTU students.
 \end{itemize}
