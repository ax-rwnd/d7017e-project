\chapter{Introduction}
\section{Background} 
At several studying programs at Luleå university of Technology, programming is a major part of the courses. Students may sometimes have some difficulty to know if they have programmed correctly until a lab supervisor controls the code written by the student. This may sometimes lead to long feedback time and the thoughts about the specific programming problems are forgotten between the feedback time. To increase the feedback time and make the programming more fun a platform for coding practice would be suitable. There are a lot of different applications on the internet but none are suitable enough for teachers and students at Luleå university of Technology. Gamification is shown to increase the activity of people, so why shouldn't gamification be suitable for programming? 

With direct feedback and with notable progress students may think that programming will be more fun. By small increments in the programming problem difficulty it's easier to succeed to the next step instead of having a large problem covering a lot of different syntax. With different difficulties every student can feel that they are challenged and receives rewards immediately. In different existing online tools it is possible for teachers to write exercises and tests where students then can solve and get feedback. 

But none of the tools found support that teachers can write exercises and have the gamification parts, they only support one of them. Many of the existing tools are complicated for teachers to implement in existing courses so it need to be simplified more for the teachers to use the platform otherwise it will not be used. The bottleneck concluded is the teachers mentioned above, they need to easily and in a few hours be able to implement a course in the platform. To fill this absence of information, interviews were conducted with teachers, students and finally a workshop was setup with teachers to try to broaden the picture of what is missing and what they need.


\section{Goals}   
The goal of the project was to find a teaching method for the students that facilitates learning for both basic and advanced programming by using gamification theories.

The purpose of gamification is to take elements from games and apply them in other areas to create an increased commitment within the applied area.
The target group is mainly students studying at Department of Computer Science, Electrical and Space Engineering  with a programming direction but also other educations within LTU that contain programming parts, for example teacher education where programming will become a central part in the future because of a revised curriculum for elementary school and high school that passed in the fall of 2017.

To reach the goal of the project, a platform that embodied these ideas was needed.
The goal was to have a platform that was easy for both users and teachers alike. Gamification was a big part of the platform but a realization that not all teachers wanted to use these elements was reached and because of this it needed to be simple to activate and disable gamification features for courses as the teacher saw fit. As such, a big goal during the development of the platform was to make features very modular.

\section{Expected result / requirements}  
The expected result was to deliver a web-based platform for learning programming. The platform should include gamification methods that stimulates the student so that they want to program more and be better at programming. The students should be able to write their code in the browser and then be able to validate if they have the correct outcome of the assignment. The platform should support progress and different levels so that every student is challenged. The platform shouldn't be limited to LTU even though LTU is the primary source that will use the platform. 

\subsection{Initial requirements} 
 \begin{itemize}
\item Full solution for writing and testing code.
\item Immediate feedback.
\item Show progress/statistics.
\item Involve gamification.
\item Support multiple programming language.
\item Teachers should be able to construct assignments.
\item Easy navigation.
\item Log in by CAS. 
\item Primary target is LTU students.
 \end{itemize}
