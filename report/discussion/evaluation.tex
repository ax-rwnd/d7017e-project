Initially, we set out to build a system that would make learning programming more fun and less tedious. This would involve gamification as a mean of picking up weaker students that would otherwise fail and give up in the face of overwhelming challenges. In addition, we wanted to remove, or at least lessen, the requirement of having personnel hired to correct labs by using automated correction.

To solve the automation, a system was developed that would create sandboxes, perform simpler tests and allow easy extensions. This way, other maintainers could add their languages of choice to the solution at a later time and select which testers they would like to use based on what languages are available and what latency they provide. Furthermore, this system was designed to scale out horizontally, meaning that if the system needs to be used by a larger amount of people, more hardware can be added to match the increased load.

Something that was not planned, but came as a consequence of how the tester was developed, is that the tester is so generic that it can be used for other projects as well. This opens possibilities for authors of similar services and games to develop new frontends quickly that make use of the service instead of reinventing it.

On the other hand, an issue that teachers of more advanced courses may have with the system is the lack of real unit tests. Without unit tests, making conclusions about application internals is nearly impossible and requires student interaction. However, an attempt was made at implementing unit testing for \emph{any} language, which showed that it could be done, see section~\ref{sec:testtypes} and section~\ref{sec:unittests_future}, but was eventually thrown away as the implementation proved both messy and hard to set up for teachers.

For the other part of helping students discover the joy of programming, like we already have, we have built a website that is both a game and a learning platform. While the game itself is not the most exciting piece of work to see the light of day, it does not spoil students into requiring games to learn either. By keeping the process of creating new game elements simple, we hope that it will be possible for others to create new elements on their own. This includes the ability to create elements that do spoil users and other bad things, but that will need to be evaluated and formally evaluated at some point.

The provided modules showcase some simpler mechanics that can be woven into the service and that expose both positive and negative traits. For instance, the progress bar \emph{may have} the negative effect of not promoting learning beyond the platform, but to prove that it does, empirical data must be gathered both from the platform and from real life. Our system lets educators and researchers gather this data by running courses with and without game elements included. This data may then be compared to the success/drop-out-rate of the students attending the course.

Unfortunately, an important part of gamification that was not possible to realize in time was group effort. A common theme discovered in the prestudy was that letting students cooperate and interact with other students introduces some interesting boons. Some angles that could be explored are further documented in section~\ref{sec:social}, but were considered too big of a risk to actually commit to at the late stage of development when the single-player elements were finished, as it might carry the hidden requirement of editing the database schema.

Fittingly, this brings us to the issue of maintainability and extensibility in the database. It was discovered, when introducing the game elements, that the database structure that had been developed could not support the level of modularity we wanted to offer. As an effect, storing custom data for game elements may require both API modifications and updates to the database schema. For the amount of game elements included in our release, it serves fine, but in extension it might not scale.

In conclusion, the service is able to provide a fun and engaging experience for students that does not necessarily spoil them into not studying. Meanwhile, educators are able to create courses that allow students to complete assignments and get rewards appropriately. This is done in a manner that can partially replace lab assignments and retain more students, all the while saving working hours that would otherwise be spent on correcting labs. Lastly, new game elements can be created by educators and researchers alike to expose data that is critical to research within the field of gamification within education.

% TODO: connect with future work and A/B testing Even without A/B testing in the system, simpler statistical measurements should % TODO: mmention that not all elements are good% Also, with the modular approach taken to the game elements, new games can be created without too much work.



