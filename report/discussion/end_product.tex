We set out to build a website that could help examiners improve their courses by using gamification. To do this, both student interaction and teacher management had to be considered to create a pleasant user-experience.

Most notably, we aimed to minimize the unnecessary feedback loop between students and lab-supervisors. By creating an extensible and scalable tool for automated testing of code, we could provide an environment that our platform---and other platforms---can use to test assignments.

By implementing a framework that would allow quick implementations of new elements, a platform for researching new gamification tools was born. After implementing a new element, it can be evaluated by looking at the completion statistics. However, there is much work that can be done on this part to empower teachers and gamification researchers with the data they need to make conclusions regarding their original research.

The solution should be fairly easy to install for the IT department and easy to use for the course examiner. Our plan of action for doing this was to deploy frontend and backend as docker containers together with a haproxy docker container for load-balancing. Configuring these would be done using files which the end-user could modify before building the docker containers, and thus set the correct settings for their setup. However, the tester environment requires docker to run, and should probably not be nested, so that needs to run on its on virtual machine. Furthermore, there wasn't enough time to actually generate a working configuration, so for the first deployment, some additional work may need to be done.
