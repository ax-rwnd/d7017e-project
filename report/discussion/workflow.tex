First off, there were 15 students in a project that was dimensioned for 7--9 students. Apart from the common issues that arise when managing very large groups, the oversized project group made members who aren't as used to voicing their opinions express themselves even less. Ironically, this is a characteristic that was discussed in the conducted interviews and the solution is usually to either help people trust one another (teambuilding), or to remove the requirement of trust (anonymization). 

Another issue was that the project specification didn't specify enough details regarding the requested product. This lead to a long prestudy wherein a lot of time was devoted to answering questions that probably could have been answered by the project owner or at least outside of the project deadline by a smaller group. To make matters worse, no work could be initiated until the prestudy was done, lest you'd want to run the risk of performing work that would be thrown away once the project goal changed. In effect, a lot of people did very little work during the four first weeks, and there was very little that could have been done about it on our part.

Delegating all inter-group communication to the leaders did not work well, probably because the needs were far too large. Most of the communication between groups happened in the project-room between developers. Another consequence of this is that developers that did not work from the project-room had trouble staying updated on the latest decisions and reporting their progress. This made it difficult to keep the developers saturated with tasks.

The project leader, who should have an overview of the development, got information about the progress from the group leaders. This information was superficial and unable to compensate for the lack of communication with the developers. This lead to a lack of technical depth in the biweekly presentations. To increase the technical content of the presentations, the group leaders had to become more involved in them. The project leader should have been more present in the project room when the developers were there to gather information about the development progress. In conclusion, having too many levels of leadership did not work well for an organization of this size.

The agile development style was well-suited to the biweekly presentations. The group was not completely satisfied with \taiga{} as a sprint planning tool. There were many features that were not used because they seemed to only increase the burden on the developers. The group leaders found that transitioning between sprints required a lot of manual work. Partly as a result of the problems with the tool, tasks gradually transitioned to being tracked on a whiteboard in the project-room. This lead to a clearer picture of the current tasks for on-site developers, but compounded the previously mentioned problems for remote developers.
