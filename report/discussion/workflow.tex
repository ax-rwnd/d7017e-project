There were 15 students in a project that was originally dimensioned for 7--9 students. Apart from the common issues that arise when managing very large groups, the oversized project group made members who aren't as used to voicing their opinions express themselves even less. This characteristic is also something that was discussed in the conducted interviews and the solution is usually to either help people trust one another (teambuilding), or to remove the requirement of trust (anonymization). 

The project specification didn't specify all details regarding the requested product. This lead to a long prestudy wherein a lot of time was devoted to answering questions that maybe could have been partially answered by the project owner if asked more thoroughly. No technical work was initiated before the prestudy was done, primarily out of the fear of performing work that would be thrown away if the project goal changed. In effect, a lot of people did very little work during the four first weeks. In hindsight a part of the group could have started looking into other technical parts that would be needed regardless of the results acquired in the prestudy.

Delegating all inter-group communication to the leaders was a challange, probably because the needs were far too large. Most of the communication between groups happened in the project-room between developers. Another consequence of this is that developers that did not work from the project-room had trouble staying updated on the latest decisions and reporting their progress. This made it difficult to keep the developers saturated with tasks. The main problem was that information was spread in a non-standardized way which resulted in some people not receiving enough information. Taiga was initially used to delegate work and issues between members but over time it became less used and a classic whiteboard became the main source of delegating tasks. Based on the experience aquired from the project, it would have been better to decide on a communication method, stick to it and enforce it as a way to minimize communication errors.

The project leader, who should have an overview of the development, got information about the progress from the group leaders. This information was superficial and unable to compensate for the lack of communication with the developers. This lead to a lack of technical depth in the biweekly presentations. To increase the technical content of the presentations, the group leaders had to become more involved in them. The project leader should have been more present in the project room when the developers were there to gather information about the development progress. In conclusion, having too many levels of leadership did not work well for an organization of this size.\\

Structuring a functional workflow with 15 members was a big challenge and an invaluable experience for all members since no one had any previous experience in working with such a large group. Even though the group faced several issues and challanges, a product was developed that each and every one could be proud to have been part of.

\subsection{Project Planning}
The agile development style was well-suited to the biweekly presentations. However, the group was not completely satisfied with \taiga{} as a sprint planning tool. There were many features that were not used because they seemed to only increase the burden on the developers. The group leaders found that transitioning between sprints required a lot of manual work. Partly as a result of the problems with the tool, tasks gradually transitioned to being tracked on a whiteboard in the project-room. This lead to a clearer picture of the current tasks for on-site developers, but compounded the previously mentioned problems for remote developers.
