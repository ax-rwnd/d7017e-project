
During the project there has been many confusions what should be done, why and how. We were 15 persons in the project although the project was dimension for approximately 7-9 persons. Non of the project members have ever worked in a such large group so it was difficult to start the project due to some people took more place than others. The projects outcome was unclear for a very long time and projects specification and requirements should be more detailed, this for help the project to start faster. The prestudy part took too long time, there was many parts to cover and required. We was told to have a workshop which outcome was good but the workshop was held when the implementation should started so the project was held for a while waiting for the result of the workshop.
\\
Since there was 15 persons in the project it become some groupings and the communicating lacked between the groups. This was a hard problem to solve and the only way to solve was to encourage everyone to sit in the classroom and talk to each other. Two different meeting methods were tested, the whole group and leaders from groups. The meeting method was changed that every group selected a leader that went to leader meeting.\\
 This had various result mainly because the meetings conclusions and discussion was sometimes shared correct to the rest of the groups and sometimes there was missing in the information, which did that the other members was misinformed about the other groups. So the conclusion is that the projects groups should be smaller, about 7-9 persons in a project is more suitable since the persons taking this course mostly doesn't have experience working in projects and more less in large groups.

\subsection{Group structure}
The group structure with group leaders for each group worked well during the first weeks but less and less information were distributed to the group member as time went by. When this problem was then identified it was to short time left on the project to change the meeting method and instead it was simply decided that everyone needed to try and keep the quality of the distributed information up like it was in the first few weeks. It should been some research in the beginning how it is suitable to divide large groups and how to communicate in best way.  \\

\subsection{Planning}
\taiga{} was evaluated during the project by some of the project members and concluded that it could have been better if it was possible to add tasks to more than one person, if it was easier to assign points to a task so that every member have equal working load based on points and extend a task to upcoming sprint. A decision was made to keep \taiga{} because the members had gained experience with it and it would be impractical to split the backlog to a new tool. There was less sprint planning in the end of the project and every member assigned tasks to them self just to keep track of progress and share with others what they were working with.

