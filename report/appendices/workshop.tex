\label{sec:workshop}
This appendix has the three workshop scenarios translated from Swedish.

\section*{Scenario 1 10 in groups + 10 min discussion}
You've recently joined a course in which you will use learn programming with a programming language that does not seem at all interesting. The teacher has given you a bunch of recommended excercises, but they do not seem to be particularly interesting. What kind of system would motivate you to perform these excercises? Think outside the box! Nothing is too crazy.

\begin{itemize}
\item Which school subject typically has this problem?
\item Do the assignments need to be corrected and presented? How?
\item What tools are needed?
\item What would be included on a list of highlights?
\end{itemize}

\section*{Scenario 2 10 in groups + 10 min discussion}
Many years have passed and you find yourself in the position of teaching that boring course which you once struggled with. Your boss runs into your office on a late Friday afternoon telling you that the course needs to be extended with more excercises. You sit down and think of a system that would motivate the students to perform these excercises, when you come to think about your old concept.

You want to make this work in the long run. What limitations do you need to set in the system now that you are a teacher? How would you implement your system in practice? Remember you might hold several courses at the same time.

\begin{itemize}
\item What is more time-demanding?
\item How much time is it allowed to take to implement and maintain the system?
\item What is too time-demanding?
\end{itemize}

\section*{Scenario 3 15 in groups + 10 min discussion}
You know have a well-functioning system. However, the students are demanding a web-tool, preferably a web-site. What does this system look like? Think web-design without the small details, try to capture the user experience.

\begin{itemize}
\item Focus on the user experience
\item How does the user give input?
\item Any other limitations at this stage?
\end{itemize}
