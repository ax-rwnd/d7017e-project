The questions asked during the interviews.\\
\begin{itemize}
 \item Do you find that student appreciate e-tools?
 \begin{itemize}
 \item Which tools do you think is needed to engage students?
 \end{itemize}
 
 \item A large problem (at least at the computer science program) is that may students are shy and doesn't dare to ask questions. Do you think that an e-tool in the classroom could simply this. e.g. anonymous feedback services? 

 
 \item Do you experience that there is different types of students that response on different types of motivation?
 \begin{itemize}
 \item If so, which trends have you experienced?
 \item Do you think that it is possible to reach more students by e-tools?
 \end{itemize}
 
 \item Are you familiar with gamification?
 \begin{itemize}
 \item How did you get it?
 \item Which pro and cons do you see?
 \item Which pitfalls have you experienced?
 \end{itemize}
 
 \item  How much time would you as a teacher be able to spend to create assignments?
 \begin{itemize}
 \item How much time do you spent on creating regular assignments?
 \end{itemize}
 
 \item What do you need available to see that this tool will work?
 \begin{itemize}
 \item Which difficulties do you see by using a e-tool, e.g. Tech.io?
 \end{itemize}
\item How can we motivate you as a teacher?
\item What would you think about if you should create a similar tool?

\end{itemize}

A summary list of the outcome of the interviews with university teachers at \LTU{} about gamification.
\begin{itemize}
 \item Mandatory assignments would increase the participation frequency but would be less fun.
 \item Public scoreboard may be demotivating.
 \item Best result could be displayed by a score.
 \item A balance between school and fun.
 \item Levels / different difficulties. 
 \item Even the weaker students should have a chance to solve assignments on the easier levels.
 \item No points or badges.
 \item Anonymous questions.
 \item Good for the lower grades.
 \item Must be easy to adapt the system into a course.
\end{itemize}