The questions asked during the interviews were as following:

\begin{itemize}
 \item Do you find that student appreciate e-tools?
 \begin{itemize}
 \item Which tools do you think is needed to engage students?
 \end{itemize}
 
 \item A large problem (at least at the computer science program) is that many students are shy and don't dare to ask questions. Do you think that an e-tool in the classroom could simply this, e.g.\ anonymous feedback services? 
 
 \item Do you experience that there are different types of students that respond on different types of motivation?
 \begin{itemize}
 \item If so, which trends have you experienced?
 \item Do you think that it is possible to reach more students by e-tools?
 \end{itemize}
 
 \item Are you familiar with gamification?
 \begin{itemize}
 \item How did you get in touch with it?
 \item Which pros and cons do you see?
 \item Which pitfalls have you experienced?
 \end{itemize}
 
 \item How much time would you as a teacher be able to spend to create assignments?
 \begin{itemize}
 \item How much time do you spent on creating regular assignments?
 \end{itemize}
 
 \item What do you need available to see that this tool will work?
 \begin{itemize}
 \item Which difficulties do you see by using a e-tool, e.g.\ \techio()?
 \end{itemize}
\item How can we motivate you as a teacher?
\item What would you think about if you should create a similar tool?

\end{itemize}

A summary of the results that arose from the the interviews:
\begin{itemize}
 \item Mandatory assignments would increase the participation frequency, but would be less fun.
 \item Public scoreboards may be demotivating.
 \item The results in a course be represented by a score.
 \item Gamified tools need to balance school and fun, or students will never get motivated to study by themselves.
 \item Levels/different difficulties can be used to avoid long feedback-loops.
 \item Even the weaker students should have a chance to solve assignments on the easier levels.
 \item Points and badges should be used with care, tools that abuse external motivation like this make students less responsive to internal motivation.
 \item Anonymous questions can be used for great good.
% \item Good for the lower grades. % What?
 \item E-tools need to be easy to adapt into courses, or they won't be used.
\end{itemize}
