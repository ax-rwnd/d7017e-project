The Gamified Programming Platform was built and tested on Debian GNU/Linux 9 (stretch) with \nodejs{} 8.6.0 and npm 5.3.0.

\section{Tester}
Tester consists of two components; Manager and Runner. Manager replies to requests from the backend and manages docker containers that run arbitrary code. Containers are used to ensure that some test $A$ does not interfere with some later test $B$ by modifying the execution environment.\\
\begin{enumerate}
    \item Clone the repo: \texttt{git clone \url{https://github.com/ax-rwnd/d7017e-project}}
    \item Change directory to the Manager folder: \texttt{cd d7017e-project/tester}
    \item Install the dependencies for the Manager: \texttt{npm i}
    \item (Optional) Select languages by adding/removing dependencies in \texttt{Makefile}. For instance the line \texttt{all: python27 python3 java c \# haskell} selects the languages Python 2.7, Python 3, Java and C, but not Haskell (since it's commented out).
    \item Run the Makefile: \texttt{make}
    \item (Optional) Set preferences for Runner in \texttt{config/default.js}. There, things like queue lengths and ports may be configured.
    \item Move back up to manager: \texttt{cd ..}
    \item Start Manager: \texttt{node server.js \{PORT\}}
\end{enumerate}

\section{Backend}
Backend is the state-managing component. It uses MongoDB to store information that it receives while processesing frontend requests and tester results.
\begin{enumerate}
\item Install och configure MongoDB.
\item Clone the repo: \texttt{git clone \url{https://github.com/ax-rwnd/d7017e-project}}
\item Change directory to backend: \texttt{cd d7017e-project/Backend}
\item Install dependencies: \texttt{npm i}
\item Configure database address/port in \texttt{Backend/config/default} and \\
\texttt{Backend/config/production}. IP/Port may differ between the files, should you want to use different databases for testing and production. To select one of these files, set the \texttt{NODE\_ENV} environment variable to \texttt{production} or \texttt{development}.
\item Start the backend daemon: \texttt{npm start}.
\item (Optional) Start backend in the foreground: \texttt{node ./bin/www}
\end{enumerate}

\section{Frontend}
Frontend is the part that the users see. It builds on Angular for UI and contacts backend for functionality.

\begin{enumerate}
    \item Clone the repo: \texttt{git clone \url{https://github.com/ax-rwnd/d7017e-project}}
    \item Change directory to frontend: \texttt{cd d7017e-project/frontend}
    \item Redirect frontend to backend: \texttt{sed -i "s/ \textbackslash (backend\_ip: \textbackslash )'.*'/ \\ \textbackslash 1'\url{https://{your\_backend}}'/" src/environments/environment.prod.ts}
    \item Tell were the global ip for frontend is: \texttt{ed -i "s/\textbackslash (frontend\_ip: \\ \textbackslash)'.*'/\textbackslash 1'\url{https://{your\_frontend}}'/" src/environments/environment.prod.ts}
    \item (Optional) Repeat step 2 and 3 for \texttt{src/environments/environment.ts}
    \item Move or link your ssl-certificates \texttt{ln -s encryption/private.key.default \\
    encryption/private.key \&\& \\
    ln -s encryption/server.crt.default encryption/server.crt}
    \item Start server: \texttt{ng serve --ssl 1 --ssl-cert ./encryption/server.crt \\
    --ssl-key ./encryption/private.key --live-reload false}
\end{enumerate}
