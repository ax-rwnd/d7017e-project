\chapter{Working methods}
\section{Sprints} 
Agile work flow was used during the project. This was done by dividing the weeks into sprints. A sprint was expected to be one week but sometimes sprints become longer due to heavy workload. Every sprint was planed before the current sprint was ended. The group leaders planned the sprints, one from each current sub group along with the project leader. Taiga.io, a sprint planning tool kept track of the backlog and different task assigned to individual group members. This showed which task was completed and which needed to be helped with. During the project some of the members evaluated the tool Taiga.io and found that it could have been better if it was possible to add tasks to more than one person, if it was easier to assign points to a task and move a task to other sprints. A decision was made to keep Taiga.io because the members had gained experience with it and it would be impractical to split the backlog to a new tool.

\section{Groups} 
All of the different task sections were split into groups. During the project it was laborated on how many number of groups would be the most efficient for the project. The working model that was found was to have three groups with 4-5 members each as long as there were enough tasks in different sections for three groups. Every week at least one time all members met and discussed the previous week and the upcoming week. This became a meeting with 15 people that ended up talking over each other which felt ineffective. This was solved by minimizing the meeting members to three-four people: the project leader along with one group leader each from the frontend, backend and tester group. The leaders were democratically chosen in each active group and their task was to tell the other group leaders what they have and will accomplish during next period. They then went to their own groups and told them about the meeting. This worked well during the first weeks but less and less information were distributed to the group member as time went by. When this problem was then identified it was to short time left on the project to change the meeting method and instead it was simply decided that everyone needed to try and keep the quality of the distributed information up like it was in the first few weeks. \\
The different groups that has been presented are:\\
 \begin{itemize}
 \item Interviews
 \item Workshop
 \item Platform
 \item Theory
 \item Frontend
 \item Backend
 \item Tester.
 \end{itemize}


\section{Roles} 
All the project members had a responsibility that the project should succeed and it was important to contribute to the extent that each member was capable of.\\
The three major roles have been assigned during the project, these are \\
\textbf{Project leader} - Niklas Fuks which had an overview of the project and its sub tasks. Planned the meetings, sprints and held presentations with demos. \\
\textbf{Groupleader and software engineers} -\\
Axel Sundbom (tester)\\
Tobias Axelsson (backend) \\
Axel Vallin (frontend) \\
\textbf{Software engineers} -\\
Linn Danielsson\\
Nils Fitinghoff\\
Mikael Hedkvist\\
David Sandström\\
Anton Jerhamre\\
Christopher Rosenvall\\
Fredrik Bostrand\\
Henrik Nilsson Harnet\\
Tim Granström\\
Rickard Nordlander\\
Anders Mikkelä\\
