% intro, what is Tester trying to do?
% we want broad language support
Tester is a tool for testing code in a variety of programming languages in a safe and isolated environment. To fulfill the needs of the GPP, Tester needs to support a  broad varierty of languages, so that the range of courses that can benefit from the platform is not limited. Tester also strives for the functionality to test code based on different merits to support the creation of varied, fun and interesting assignments.

\subsection{Sandboxing}
%TODO ref to docker
An important feature of the Tester is that the code is tested in an isolated enviroment. The main reason behind doing this is that we are executing potentially malicious code. Allowing such code to be executed without isolation is bound to cause service disruptions or information leaks. 

To achieve isolation all code testing takes place in so called Docker containers equipped with the tools necessary to compile and run the code. A container is a virtualization (simulation) of an operating-system with a separate user-space. What this means is that programs running inside the container can only access the contents and devices that are assigned to the container. By running the tests in Docker containers, any malicious code cannot gain access to information it shouldn't have access to or cause any disruptions to the service, since such code will at most crash the container which has no long-lasting effects on the service offered by Tester.
\subsection{Manager}
% TODO: add discussion of different ways of managing load and what we chose
% (Start containers on demand? Keep a pool of containers ready? Predict load or just react? Allocate different amounts of resources for different languages?)
\subsection{Test types}
% unit tests are hard to write
% I/O tests are language-independent which is good if we want broad language support

\subsection{Extensibility}
In order to achieve broad language support, it needs to be easy to add new languages. Two steps are necessary to add support for a new language:

\begin{itemize}
\item The manager has a makefile that produces docker images for each supported language. The makefile needs to be extended with a new target that installs language-specific dependencies.

\item The Node instance running in the container needs to have a module for each supported language. Each language module exports two functions: \texttt{prepare}, which produces a file that can be run with the \texttt{run} function. The \texttt{prepare} step is useful for compiled languages, where a binary must be produced before running the program. Some languages, such as python, are run directly in the interpreter. By making preparation a separate step, the program does not need to be recompiled for each test.
\end{itemize}

\subsection{Language availability fallback}
